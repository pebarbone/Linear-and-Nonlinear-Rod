\documentclass[12pt,american]{article}
\usepackage[T1]{fontenc}
\usepackage[latin9]{inputenc}
\usepackage[letterpaper]{geometry}
\geometry{verbose,tmargin=1in,bmargin=1in,lmargin=1in,rmargin=1in}
\usepackage{textcomp}
\usepackage{amstext}
\usepackage{amssymb}
\usepackage{amsmath}
\usepackage{babel}
\usepackage{tcolorbox}

%---------------------------------------------------------
%  Toggle to label equations 
%
\def \plabel#1 {\label{#1} }
% \def \plabel#1 {\label{#1} \qquad{#1} }
%
%---------------------------------------------------------
\def\beq{\begin{equation}}
\def\eeq{\end{equation}}
\def\bea{\begin{eqnarray}}
\def\eea{\end{eqnarray}}

\def \be{\mbox{\boldmath{$e$}}}
\def \u{\mbox{\boldmath{$u$}}}
\def \bx{\mbox{\boldmath{$x$}}}
\def \ep{\epsilon}
\def \sigo{\sigma_{o}}
\def \pie{\Pi_{Elastic}}
\def \pit{\Pi_{Total}}
\def \cS{{\cal{S}}}
\def \cV{{\cal{V}}}
\def \us{{u^{*}}}

\begin{document}

\title{Nonlinear elastic rod}


\author{Andre, Paul, Elise}


\date{\today}

\maketitle

\tableofcontents
\newpage


\section{Introduction}

We derive equations of equilibrium for uniaxial deformation of an elastic rod.  
We approach the problem from three perspectives:  strong form (direct equilibrium equation), weak form, 
and minimization of 
the total potential energy.  

We eventually consider the material behavior of the rod as nonlinear, but assume the strains 
are sufficiently small to neglect geometric nonlinearity.  To begin with, however, 
we start with linear elastic material assumption. 

\section{Strong formulation}
\subsection{Fundamental equations}

Here we let $u(x)$ represent the axial displacement of a particle whose initial position is $x$, 
$\ep(x)$ is the axial strain in the rod at location $x$, and $\sigma(x)$ is the stress acting on the 
cross section of the rod at 
location $x$.  $f(x)$ is a body force (i.e.\ a force per unit volume) assumed to be known in the rod.  
Here we focus on the special case that the rod has uniform cross sectional area, $A$, and zero traction along 
its length.  

\bigskip
\noindent
{\em Equilibrium equation:}
\beq
\frac{d\sigma}{dx}  + f = 0 
\plabel{equil}
\eeq
%{\em Linear constitutive equation:}
%\beq
%\sigma = E \epsilon
%\plabel{linear-sig-ep}
%\eeq
{\em Constitutive equation:}
\beq
\sigma = \Sigma(\ep) = E \ep 
\plabel{sig-ep}
\eeq
{\em Kinematics equation:}
\beq
\ep = \frac{d u}{dx} 
\plabel{u-ep}
\eeq

\begin{tcolorbox}[title= Exercise]
Derive equation (\ref{equil}). 
\end{tcolorbox}

\subsection{An example $1D$ boundary value problem} 

We now consider a rod that has length $L$.  It is fixed at $x=0$ and is acted upon at $x=L$ by a force $P$.  
We let $\sigo = P/A$.   A body force $f(x)$ is known.    Therefore, in addition to equations (\ref{equil}-\ref{u-ep}), 
we need to add the boundary conditions: 
\bea
u(0) &=& u_o
\plabel{bc1} \\ 
\sigma(L) &=& P/A = \sigo .
\plabel{bc2}
\eea 

\begin{tcolorbox}[title= Strong Form 1]{
% \fbox{
Given $u_o$, $\sigo$, $E$,  and $f(x)$,  for $0 < x < L$, 
find 
$u(x)$, $\ep(x)$, $\sigma(x)$ that satisfy:

\bea
\frac{d\sigma}{dx}  + f &=& 0 \\  
\sigma &=& E \ep \\ 
\ep &=& \frac{d u}{dx} \\
u(0) &=& u_o \\ 
\sigma(L) &=& \sigo .
\eea 
}
\end{tcolorbox}

This box summarizes all our fundamental equations and boundary conditions.  The unknowns are $u, \ep, \sigma$.  
To solve for these unknowns, we have two first-order odes, and one algebraic equation (\ref{sig-ep}).  
If we were to set about to solving this problem, the first thing that we might do is use equations (\ref{sig-ep}) 
and (\ref{u-ep}) to eliminate the unknowns $\sigma$ and $\ep$ in favor of $u(x)$.  This would then give us: 


\begin{tcolorbox}[title= Strong Form 2]{
% \fbox{
Given $u_o$, $\sigo$, $E$, and $f(x)$,  for $0 < x < L$, 
find 
$u(x)$, that satisfies:

\bea
\frac{d}{dx} \left(E \frac{du}{dx} \right) + f &=& 0
\plabel{11} \\  
%\sigma &=& \Sigma(\ep) \\ 
%\ep &=& \frac{d u}{dx} \\
u(0) &=& u_o 
\plabel{12} \\  
\left. E \frac{du}{dx} \right|_{x=L} &=& \sigo .
\plabel{13} 
\eea 
}
\end{tcolorbox}
Here we have just the one unknown function, $u(x)$ and one differential equation.  

These problems are equivalent, but they are different.  
One has three unknown functions; the other one unknown function.
One has two first order differential equations; the other had one second order differential equation. 
One yields all quantities of potential interest (i.e. $u, \ep, \sigma$); the other yields only $u$.  
If one uses the Strong Form 2 and wishes to know the stress, then the stress must be computed as a 
{\em post-process}.  Generally speaking, if one knows any of the functions, $u, \ep, \sigma$, then 
the problem is considered to be solved.  The others can be evaluated {\em relatively} 
easily from (\ref{sig-ep}) and/or (\ref{u-ep}).  

\section{Minimum potential energy}

Here we reformulate the problem using the principle of minimum potential energy.  

\subsection{Fundamental equations} 

\subsubsection{Stored elastic potential energy}

Let $W$ denote the elastic energy density, i.e.\ the energy per unit volume, contained in a deformed solid.  
(Normally we think of $W = W(\ep)$ to be a pure function of strain (i.e.\ it depends {\em only} on strain).  
One definition of an elastic material is that $W = W(\ep)$.)  For a linear elastic material in 
uniaxial tension, $W = \frac{1}{2} E \ep^2$. 

\bigskip
\noindent
{\em Elastic potential energy:}
\bea
\pie[u] &=& \int_V W(\ep(u)) dV  \\ 
	&=& \int_0^L W(\ep(u)) A\, dx 
\plabel{pie}
\eea
Note that in (\ref{pie}), we think of $\ep$ as determined by $u(x)$ through equation (\ref{u-ep}).  

\subsubsection{Work done by loading} 

The sum of all work done by all externally applied loads is: 

\bigskip
\noindent
{\em External work:}
\bea
\ell[u] &=& \int_V f(x) \cdot u(x) dV  + P \cdot u(L)  \\
	&=& \int_0^L f(x) \cdot u(x) A\, dx  + P \cdot u(L) 
\plabel{ell}
\eea
Again, 
Note that here $V = (0,L) \times A$, and $dV = A\, dx$. 
The work done by these loads comes at the loss of some energy.  We shall assume that this work 
represents energy lost by some conservative loading machine, so that 
\beq
\Pi_{Load}[u] = - \ell[u]. 
\eeq
That is, work done on the rod is the energy lost by $\Pi_{Load}$.  

\subsubsection{Total potential energy}

The total potential energy is, therefore, $\pit = \pie + \Pi_{Load}$, or 

\bigskip
\noindent
{\em Total potential energy:}  
\bea
\pit[u] &=& \pie[u] - \ell[u] \\ 
	&=& \int_0^L W(\ep(u))  - f(x) \cdot u(x) A\, dx  - P \cdot u(L) 
\eea

\subsection{Principle of minimum potential energy} 

\begin{tcolorbox}[title= Principle of minimum potential energy - wordy]
Of all admissible functions, $u(x)$, the equilibrium 
displacement field is given by the $u(x)$ that minimizes $\pit[u]$.
\end{tcolorbox}

\subsubsection{Admissible displacement fields} 

Admissible displacement fields must satisfy an appriate degree of continuity and 
satisfy our {\em essential} boundary conditions.  In our example boundary value 
problem, the essential boundary condition is (\ref{bc1}).  The appropriate degree of 
continuity in $1D$ is that the functions be at least continuous.  
% An alternative way 
% of prescribing that the function, $u(x)$ be continuous in $1D$ is to require: 
% \beq
% \int_0^L \left(\frac{du}{dx}\right)^2 \dx < \infty. 
% \eeq
The set of continuous functions defined for $0 \le x \le L$ 
is denoted $C^0[0,L]$.  
To say that 
$u(x)$ is continuous is equivalent to saying that 
$u(x)$ is in the set $C^0$, or in math-speak, $u(x) \in C^0[0,L]$. 
\footnote{The superscript $0$ indicates how many derivatives of 
$u(x)$ are required to be continuous.  Thus $C^3[0,L]$ is the set of all functions 
whose third derivative (and therefore second and first and zeroth) 
is continuous between $0 \le x \le L$.}

Therefore we introduce the set $\cS$ of admissible displacement fields, $u(x)$: 
\beq
\cS = \left\{ u(x) | u(x)\in C^0[0,L];\  u(0) = u_o. \right\}
\plabel{S}
\eeq
This reads, ``$\cS$ is the set of all $u(x)$ such that $u(x)$ is continuous and $u(0) = 0$.''


\begin{tcolorbox}[title= Principle of minimum potential energy - mathy]
The equilibrium displacement field, $\us$, is the minimizer over all 
$u(x) \in \cS$ of: 
\bea 
\pit[u] &=& \int_0^L \frac{1}{2} E \left(\frac{du}{dx} \right)^2  - f(x) \cdot u(x) A\, dx  - P \cdot u(L),  
\plabel{pit}
\\ 
\mbox{where} \qquad 
\cS &=& \left\{ u(x) | u(x)\in C^0[0,L];\  u(0) = u_o. \right\}
\eea
\end{tcolorbox}


\subsubsection{Admissible variation fields} 

Pick any $u(x) \in \cS$.  Then $v(x)$ is an {\em admissible variation} 
if (and only if) $u(x) + v(x) \in \cS$.  We let $\cV$ denote the set of all 
admissible variations.  

\begin{tcolorbox}[title=Exercise:]
Use the definition given to show that 
\beq
\cV = \left\{ v(x) | v(x)\in C^0[0,L];\  v(0) = u_o. \right\}
\plabel{V}
\eeq
\end{tcolorbox}


\subsection{Euler Lagrange Equations:  The connection between the strong form 
and the principle of minimum potential energy} 

We let $\us\in \cS$ denote the minimizer of $\pit[u]$.  
Our goal here is to show that $\us$ satisfies Strong Form 2.  

We begin by choosing an arbitrary function $v(x) \in \cV$.  
Once we pick that function, it's picked.  It doesn't change.  
To say that it's arbitrary means that it doesn't matter which one we pick, 
but we have to pick one.  In this example, we can imagine it's $v(x) = 8x$.  

We next define the function $F(\alpha)$, where $\alpha$ is an arbitrary 
scalar, and 
\beq
F(\alpha) = \pit[\us + \alpha v]. 
\plabel{100}
\eeq
We note that since $\us$ minimizes $\pit[u]$, 
then $F(\alpha)$ is minimum at $\alpha = 0$: 
\beq
F(\alpha) = \pit[\us + \alpha v] \ge \pit[\us ]  = F(0). 
\plabel{101}
\eeq
Since $\alpha =0$ is a minimum of $F(\alpha)$, then the slope of $F$ is zero at that 
point.  Hence 
$\left.\frac{dF}{d\alpha}\right|_{\alpha = 0} = 0$.  Computing gives us: 
\bea
\left.\frac{dF}{d\alpha}\right|_{\alpha = 0} 
 	&=&
\left.\frac{d}{d\alpha}\right|_{\alpha = 0} 
\int_0^L \frac{1}{2} E 
\left(\frac{d\us}{dx}  
+ \alpha \frac{dv}{dx}  \right)^2  
- f(x) \cdot (\us + \alpha v(x)) A\, dx  
% \nonumber \\
% && \qquad \qquad 
- P \cdot (u(L) + \alpha v(L))
\nonumber \\
 	&=&
\left.\left\{ 
\int_0^L E 
\left(\frac{d\us}{dx}  
+ \alpha \frac{dv}{dx}  \right) 
\frac{dv}{dx}  
- f(x) \cdot (v(x)) A\, dx  - P \cdot (v(L))
\right\}
\right|_{\alpha = 0} 
\\
 	&=&
\int_0^L E 
% \left(
\frac{d\us}{dx}  
 % \right) 
\frac{dv}{dx}  
- f(x) \cdot v(x) A\, dx  - P \cdot v(L)
\plabel{102} \\ 
 	&=& 0 \qquad \forall v(x) \in \cV 
\plabel{103}
\eea
To see the connection between (\ref{102}) and (\ref{11}), 
we use integration by parts on the first term in (\ref{102}).  

Integration by parts is really the product rule of differentiation backwards. 
Remembering this helps when applying it in multi-dimensional contexts.  
Therefore, in order to integrate (\ref{102}) by parts, we first rewrite 
the product of derivatives using the product rule: 
\beq
\frac{d}{dx}  \left( E \frac{d\us}{dx}  v(x) \right) 
% &=& 
=
E \frac{d\us}{dx}  
\frac{dv}{dx}  
+ 
v(x) \frac{d}{dx}  \left( E \frac{d\us}{dx}  \right)
\eeq
Therefore, 
\bea
\int_0^L E 
\frac{d\us}{dx}  
\frac{dv}{dx}  
A\, dx  
 	&=& 
\int_0^L E 
\frac{d}{dx}  \left( E \frac{d\us}{dx}  v(x) \right) 
- 
v(x) \frac{d}{dx}  \left( E \frac{d\us}{dx}  \right)
A\, dx  
\\
 	&=& 
\int_0^L E 
\frac{d}{dx}  \left( E \frac{d\us}{dx}  v(x) \right) 
A\, dx  
- 
\int_0^L E 
v(x) \frac{d}{dx}  \left( E \frac{d\us}{dx}  \right)
A\, dx  
\\
 	&=& 
\left.\left( E \frac{d\us}{dx}  v(x) \right) 
A\right|_{x=0}^{x=L} 
- 
\int_0^L E 
v(x) \frac{d}{dx}  \left( E \frac{d\us}{dx}  \right)
A\, dx  
\\
\mbox{since\ $v \in \cV$:} \qquad  	&=& 
\left.\left( E \frac{d\us}{dx}
\right) A \right|_{x=L} 
  v(L) 
- 
\int_0^L E 
v(x) \frac{d}{dx}  \left( E \frac{d\us}{dx}  \right)
A\, dx  
\plabel{104}
\eea


\begin{tcolorbox}[title= Exercise]
Get (\ref{104}) from the line above.  When you know how to do it, it's one line.  
\end{tcolorbox}

We now use (\ref{104}) to rewrite equation (\ref{102}) in the form: 
\beq
- \int_0^L 
v(x) 
\left[
\frac{d}{dx}  \left( E \frac{d\us}{dx}  \right)
+ f(x) \right] A\, dx  
+\left[ \left.\left( E \frac{d\us}{dx}
\right) \right|_{x=L} A
- P \right] v(L)
 	= 0 \qquad \forall v(x) \in \cV 
\plabel{105}
\eeq
Since (\ref{105}) holds for any choice of $v(x) \in \cV$, then we can conclude:
\bea
\frac{d}{dx}  \left( E \frac{d\us}{dx}  \right) + f(x)  
&=& 0 
\plabel{106} \\ 
\left.\left( E \frac{d\us}{dx}
\right) \right|_{x=L} 
&=& P . 
\plabel{107} 
\eea
Equations (\ref{106}) and (\ref{107}) are the Euler-Lagrange equations resulting from 
minimizing the functional $\pit[u]$.  

We see that $\us(x)$ 
satisfies differential equation (\ref{106}), which is the same differential 
equation as (\ref{11}).  Furthermore, $\us(x)$ satisfies boundary condition 
(\ref{107}), which is the same boundary condition as (\ref{13}).  
Since $\us(x)$ is required to be in the set $\cS$, then $\us$ also satisfies (\ref{12}).  
Therefore $\us$ satisfies equations (\ref{11}, \ref{12}, \ref{13}) of Strong Form 2.  
Since the solution of Strong Form 2 is unique, then its solution also minimizes 
$\pit[u]$ over all functions in $\cS$.  

In the process of minimizing $\pi[u]$, we found that boundary condition 
(\ref{107}) (or (\ref{13})) resulted naturally as a consequence of the minimization.  By contrast, 
boundary condition (\ref{12}) is an essential part of the formulation of the minimization problem. 
Without boundary condition (\ref{12}), the minimization problem wouldn't make sense; there would be 
no minimum.  Therefore, (\ref{12}) is called an {\em essential boundary condition}, while (\ref{13}) 
is called a {\em natural boundary condition}. 



\begin{tcolorbox}[title= Summary] 
The function $\us(x)\in \cS$ minimizes the total potential energy $\pit[u]$, compared to all 
other functions in $\cS$.  The minimizing function $\us$ 
satisfies equations (\ref{11}, \ref{12}, \ref{13}) of Strong Form 2, 
and likewise, the solution of Strong Form 2 minimizes 
$\pit[u]$. 
\end{tcolorbox}


\section{Weak formulation}

\subsection{Obtained from minimum potential energy}

Above we showed that Strong Form 2 follows from equation (\ref{102}).  
We divide (\ref{102}) through by the constant area $A$ and write that result here 
for convenience: 
\beq
\int_0^L E 
\frac{d\us}{dx}  
\frac{dv}{dx}  
- f(x) \cdot v(x) \, dx  - \sigo \cdot v(L)
 	= 0 \qquad \forall v(x) \in \cV 
\plabel{weak1}
\eeq
Equation (\ref{weak1}) is known as the weak form the boundary value problem.  More precisely, 
the weak form of the boundary value problem is:  

\begin{tcolorbox}[title= Weak Form] 
Given $u_o$, $\sigo$, $E$,  and $f(x)$,  for $0 < x < L$, 
find $u(x) \in \cS$ such that for all $v(x) \in \cV$, 
\beq
\int_0^L E 
\frac{d\us}{dx}  
\frac{dv}{dx}  
- f(x) \cdot v(x) \, dx  - \sigo \cdot v(L)
 	= 0 
\plabel{weak}
\eeq
\end{tcolorbox}





\subsection{Derivation from Strong Form 2}

\begin{tcolorbox}[title= Exercise: Derive the weak form from the strong form] 
Let $v \in \cV$.  Compute $\int_0^L (\ref{11})\,  v(x) dx$.  
Integrate by parts and simplify to derive (\ref{weak}).  
In the process of getting (\ref{weak}), you will need to use (\ref{12}) and (\ref{13}) 
various steps.  Since you use (\ref{11}, \ref{12}, \ref{13}), this implies that the 
solution of Strong Form 2 also satisfies Weak Form for $\u \in \cS$.  
Can you identify where each of these eqns is used? 
(\ref{12}) is subtle.
\end{tcolorbox}




\section{A mixed weak form}

\subsection{Derivation from Strong Form 1}

\subsection{Derivation from a mixed variational principle}





\end{document}













